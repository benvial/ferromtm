
%--some needed packages--------------------------------------------------------
\usepackage[english]{babel}  % language
% \usepackage[utf8]{inputenc}
\usepackage{amsmath,amssymb}
\usepackage{lipsum}
\usepackage[absolute,showboxes,overlay]{textpos}  %% textblocks
\TPshowboxesfalse  % show textblock borders


%%%%%%%%%%%%%%%%%% COLORS %%%%%%%%%%%%%%%%%%%%%%%%
\usepackage{xcolor}


\definecolor{fresnellightblue}{HTML}{424242}
\definecolor{fresnelmiddleblue}{HTML}{FCA822}
\definecolor{fresneldarkblue}{HTML}{D6A977}
\definecolor{fresnelblue}{HTML}{9E5B5B}%{974E4E}
\definecolor{fresnelorange}{HTML}{FFFFFF}
\definecolor{fresnellightgray}{HTML}{FFFFFF}
\definecolor{fresnelgray}{HTML}{656263}
\definecolor{fresnelred}{HTML}{D64937}
\definecolor{fresnelgreen}{HTML}{95C11F}
\definecolor{fresneldarkgray}{HTML}{9E5B5B}%    {634A49}
\definecolor{fresnelblack}{HTML}{292A33}
\definecolor{fresnelgreenblue}{HTML}{9E5B5B}%{2F8A82}
\definecolor{fresnelcaption}{HTML}{4E4D52}%{2F8A82}



\definecolor{benchcolor1}{HTML}{9E5B5B}  %%% red %% frame title and rule
\definecolor{benchcolor2}{HTML}{6FBD90}  %%  green
\definecolor{benchcolor3}{HTML}{424242}  %%% dark grey/ black  %%normal text
\definecolor{benchcolor4}{HTML}{03658C}  %%   blue
\definecolor{benchcolor5}{HTML}{E05A00}   %% orange
\definecolor{benchcolor6}{HTML}{424242}  %%%% dark grey/ black
\definecolor{benchcolor7}{HTML}{179D9D}  %%%% green blue
\definecolor{benchcolor8}{HTML}{B6855A}  %%%% brown
\definecolor{benchcolor9}{HTML}{9F61A8}  %%%% purple
\definecolor{benchcolor10}{HTML}{E09EDE}  %%%% pink
\definecolor{benchcolor11}{HTML}{A1A1A1}  %%%% light gray
\definecolor{benchcolor12}{HTML}{D0D0D0}  %%%% lighter gray


%%%%%%%%%%%%%%   Fonts %%%%%%%%%%%
%%%%%%%%   pgfplots and tikz for awesome drawings %%%%%%%%%%%
\usepackage{pgfplots}
  \pgfplotsset{compat=newest}
  \pgfplotsset{plot coordinates/math parser=false}
  \usepgfplotslibrary{patchplots}
 \usepackage{tikz}
  \usetikzlibrary{%
      decorations.pathreplacing,%
      decorations.pathmorphing,%
  shapes%
  }
\usetikzlibrary{plotmarks}





%%%%%%%%   Fonts %%%%%%%%%%%
 \usepackage[math]{iwona}% math font
%  \usepackage{arev}
 % \usepackage{cmbright}

 \DeclareMathSizes{9.8}{17}{7}{7}

\usepackage[no-math]{fontspec}
\usepackage{xunicode} %Unicode extras!
\usepackage{xltxtra}  %Fixes
% \setmainfont[Ligatures=TeX]{Avenir Next Medium}
%\usepackage{mathastext}
 \usepackage{fontawesome}
%
% %
\setmainfont[BoldFont = {Open Sans Bold},
             ItalicFont = {Open Sans Italic},
             BoldItalicFont = {Open Sans Bold Italic}]
            {Open Sans}

   % \newfontfamily{\romanfont}[Scale=1]{Garogier}
 \newfontfamily{\semiboldfont}[Scale=1.3]{Oswald Regular}
 \newfontfamily{\condensedfont}[Scale=1.3]{Oswald Regular}
 \newfontfamily{\blackfont}[Scale=1.3]{Oswald Regular}

% \newfontfamily{\FA}{FontAwesome}
% \def\twitter{{\FA \faTwitter}}

\newcommand{\ie}{i.\,e.~}
\newcommand{\Ie}{I.\,e.~}
\newcommand{\eg}{e.\,g.~}
\newcommand{\Eg}{E.\,g.~}
% \newcommand{\e}{\mathrm{e}}
\newcommand{\ic}{i}

\newcommand*\conj[1]{
  \vbox{
  \hrule height 0.3ptword
  \kern0.5ex
  \hbox{
  \kern-0.4em
  \ifmmode#1\else\ensuremath{#1}\fi
   \kern-0.em
  }
 }
}
\renewcommand{\B}{\boldsymbol}
% \newcommand{\tens}[1]{\B{#1}}
% \newcommand{\tens}[1]{\underline{\underline{#1}}}
\newcommand{\tens}[1]{\B{#1}}
\newcommand{\re}{\mathrm{Re}}
\newcommand{\im}{\mathrm{Im}}
\newcommand{\grad}{\B{\mathrm{\nabla}}}
\renewcommand{\div}{\B{\mathrm{\nabla\cdotp}}}
\newcommand{\ddroit}{\mathrm{d}}
\newcommand{\epsf}{\varepsilon^{\rm f}}
\newcommand{\epsftens}{\tens{\varepsilon}^{\rm f}}
\newcommand{\epstens}{\tens{\varepsilon}}
\newcommand{\epsd}{\varepsilon^{\rm d}}
\newcommand{\epsvac}{\varepsilon_{0}}
\newcommand{\epshom}{\tilde{\epstens}}
\newcommand{\fig}[1]{Fig.~(\ref{#1})}
\newcommand{\equ}[1]{Eq.~(\ref{#1})}

\newcommand{\xitens}{\tens{\xi}}
\newcommand{\xihom}{\tilde{\xitens}}

\usepackage{mdframed}
% \newenvironment<>{posterblock}[2]{%
%  \begin{actionenv}#3%
%  \def\insertblocktitle{\leftskip=20pt\rightskip=20pt \\ { \vspace{20pt} \textbf{\huge #1} }
%  \\ \vspace{10pt} \textcolor{fresnelblue}{\textnormal{\LARGE #2}} \vspace{20pt}}%
%  \par%
%  \usebeamertemplate{block begin}\leftskip=20pt\rightskip=20pt}
%  {\par\vspace{20pt}\usebeamertemplate{block end}
%  \end{actionenv}}

 \newenvironment{posterblock}[2]{%
  \begin{mdframed}[innerleftmargin =0.85em,innerrightmargin =0.85em,
  innerbottommargin=0.85em,innertopmargin=0.85em,
  linewidth=0pt,linecolor=#2,backgroundcolor=#2]
    {\blackfont \color{fresnelorange} \textbf{\Large \uppercase{#1}}}

%  \\ \vspace{10pt} \textcolor{fresnelblue}{\textnormal{\LARGE #2}} \vspace{20pt}}%
  \end{mdframed}
  \begin{mdframed}[innerleftmargin =1cm,innerrightmargin =1cm,
  innerbottommargin=1em,innertopmargin=1em,
  topline=false,linewidth=8pt,linecolor=white,backgroundcolor=white]
  \color{fresnelblack}
}{
\end{mdframed}
}



   \newcommand{\myblocktitle}[2]{
    {\color{#2}\textbf{\semiboldfont\Large\uppercase{#1}}\\ \vspace{-0.5em} }
    \textcolor{#2}{\rule{\columnwidth}{8pt}}\\
    \vspace{0.2em}
%     {\textcolor{fresnelblue}{\textnormal{\LARGE #2}}}\\
%     \vspace{0.5em}
}



\newcommand{\mysubtitle}[1]{\vspace*{0.5em}
{\semiboldfont \textcolor{benchcolor2}{{\textbf{#1}}}}\vspace*{0.2em}\\
}


%%%   biblio, biblatex, biber   %%%%%%%%%%%%%%
\usepackage{csquotes}
\usepackage[style=numeric-comp,citetracker=true,sorting=none,natbib=true,backend=biber,
abbreviate = true,
maxcitenames=3,
firstinits=true,
maxnames=1,
doi=false,
    isbn=false,bibencoding=utf8,
    url=false]{biblatex}

\renewcommand{\bibfont}{\small}
\addbibresource{../paper/biblio.bib}
  \urlstyle{same}
 \renewbibmacro{in:}{}
% \AtEveryBibitem{\clearfield{title}}
 \AtEveryBibitem{\clearfield{note}}
 \AtEveryBibitem{\clearfield{language}}
% % % {\DeclareFieldFormat*{volume}{\textbf{#1}}}
 \AtEveryBibitem{\DeclareFieldFormat{number}{\textbf{#1}}}
 \AtEveryBibitem{\DeclareFieldFormat*{journaltitle}{{\itshape {#1}}}}
 \AtEveryBibitem{\DeclareFieldFormat*{volume}{\bfseries{#1}}}
  \AtEveryBibitem{\DeclareFieldFormat*{edition}{{#1\addnbspace Ed.}}}


  \usepackage[orientation=portrait,
              size=a0,          % poster size
              scale=1.0         % font scale factor
             ]{beamerposter}    % beamer in poster size



             \usefonttheme{serif}
  \usefonttheme{professionalfonts}% use own font handling

  %%%%%%%%%%%%%%%%%  Beamer stuff %%%%%%%%%%%%%%%%%%%%%

  % \setbeamerfont{block title}{series=\bfseries,size=\huge}
  %\setbeamerfont{title in head/foot}{size={},series=\normalfont}
  \setbeamercolor{section in head/foot}{bg=fresnelblue, fg=fresnelblue}
  \setbeamercolor{subsection in head/foot}{bg=fresnelblue, fg=fresnelblue}
  \setbeamercolor*{block title}{bg=white, fg=black}
  \setbeamercolor*{block body}{bg=white, fg=black}
  \setbeamercolor*{block title example}{bg=fresnelgreen,fg=white}
  \setbeamercolor*{block body example}{bg=, fg=fresnelgreen}
  \setbeamercolor*{block title alerted}{bg=fresnelorange,fg=white}
  \setbeamercolor*{block body alerted}{bg=, fg=fresnelorange}

  \setbeamerfont{caption}{shape=\itshape,size=\small}
  \setbeamercolor{caption}{fg=fresnelcaption}
  \setbeamertemplate{caption}{\usebeamercolor{caption}\usebeamerfont{caption}\raggedright\insertcaption\par}

  \setbeamercolor{bibliography entry title}{bg=white, fg=fresnelblack}%
  \setbeamercolor{bibliography entry location}{bg=white, fg=fresnelblack}%
  \setbeamercolor{bibliography entry note}{bg=white, fg=fresnelblack}%
  \setbeamercolor{bibliography entry author}{bg=white, fg=fresnelblack}%
  \setbeamercolor{bibliography  item}{bg=white, fg=fresnelblack}%
  \setbeamertemplate{bibliography item}[text]
  \setbeamertemplate{navigation symbols}{}


  \setbeamertemplate{enumerate items}[circle]
  \setbeamercolor*{enumerate}{fg=fresnelorange,bg=white}
  \setbeamertemplate{enumerate subitem}{\alph{enumii}.}
  \setbeamercolor*{enumerate subitem}{fg=fresnelgreen,bg=white}
   \setbeamercolor{item projected}{bg=fresnelorange,fg=white}
  \setbeamercolor{enumerate subsubitem}{bg=,fg=fresnelblue}
   \setbeamertemplate{enumerate subsubitem}{\roman{enumiii}.}



   \let\olditem\item
  \renewcommand{\item}{\olditem  \color{fresnelblack}}


  \setbeamertemplate{itemize items}[square]
    \setbeamertemplate{itemize items}{%
       \tiny\raise1.5pt\hbox{\color{fresnelblue}$\blacksquare$}%
  }
   \setbeamertemplate{itemize subitem}{%
      \tiny\raise1.5pt\hbox{\color{fresnelgreen}$\blacktriangleright$}%
  }
   \setbeamertemplate{itemize subsubitem}{%
      \tiny\raise1.5pt\hbox{\color{fresnelblue}$\bullet$}%
  }

  \addtolength{\leftmargini}{2.5em}



  \graphicspath{
  {../data/figures/}{./figs/}}


  %%%%%%%%%%%%%%%%% %%%%%%%%%%%%%%%%%%%%%%%%%%%%%%%%%%%%%% %%%%%%%%%%%%%%%%%%%%%

  % \usebackgroundtemplate{\includegraphics[width=1.048\paperwidth]{bg3.jpg}}
  \setbeamertemplate{background}{
  {\leavevmode
    \begin{tikzpicture}
    \useasboundingbox (0,0) rectangle(\the\paperwidth,\the\paperheight);
     % \fill[color=fresnellightgray] (0,0) rectangle (111,122);
      \fill[color=fresneldarkgray] (0,108) rectangle (111,122);
      \fill[color=fresnellightblue] (0,98) rectangle (111,108);
      \fill[color=benchcolor3] (0,0) rectangle (111,7);
  %     \fill[color=fresnelblue] (0,103) rectangle (111,104);
  %     \fill[color=fresnelblue] (0,7) rectangle (111,8);
  % \fill[color=fresnellightblue] (0,6) rectangle (111,7);
  %  \fill[color=fresnellightgray] (40,8) rectangle (80,20);
    \end{tikzpicture}
  }
  }

  \setbeamersize{text margin left=6cm,text margin right=2cm}
  \setbeamertemplate{footline}{\vspace{6cm}}
  \setbeamertemplate{headline}{\vspace{35cm}}
