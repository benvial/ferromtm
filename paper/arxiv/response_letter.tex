
\documentclass[%
 aip,
 amsmath,amssymb,
%preprint,%
 reprint,%
]{revtex4-1}

\usepackage{xcolor}
\definecolor{correction}{HTML}{de5e5e}
\definecolor{reply}{HTML}{249150}
\newcommand{\co}[1]{\textcolor{correction}{#1}}
\newcommand{\rep}[1]{\textcolor{reply}{\textit{Reply:} #1}}

\begin{document}

%%%% Article title to be placed here
\title{Reponse letter for the manuscript "Enhanced tunability in ferroelectric metamaterials through local field enhancement and the effect of disorder"}
\author{Benjamin Vial and Yang Hao}

%%%%%%%%% Insert author address here
\affiliation{School of Engineering and Computer Science, Queen Mary, University of London, London, E1 4NS, United Kingdom}


%

%%%% Abstract text to be placed here %%%%%%%%%%%%
\begin{abstract}
 We would like first to thank the reviewers for their constructive comments and remarks that will surely
 improve the overall quality and clarity of the article. We address here point by point the reviewers concerns and indicate how the manuscript has been revised, with the hope that those changes will satisfy the standards of publication in \textit{Journal of Applied Physics.}
\end{abstract}
%%%%%%%%%%%%%%%%%%%%%%%%%%%

\maketitle


\section{Reviewer \#1}
\subsection{Evaluations}
\begin{itemize}
  \item Does the manuscript present original and timely results that significantly advance the knowledge in applied physics: Yes
  \item Does the manuscript report on convincing and rigorous data methods and analysis: Yes
  \item Is the manuscript clearly written in correct English well organized and free from ambiguities: Yes
  \item Is the title descriptive of the contents concise interesting and free of acronyms: No
  \item Does the abstract adequately and clearly describe the contents (problem approach findings) of the paper: No
  \item Are the figures in the manuscript necessary adequate well presented and clearly labeled: No
  \item Is the reference list appropriate: No

\end{itemize}


\subsection{Remarks to author(s)}

In this paper the authors present a computational study of dielectric tunability of ferroelectric
 metamaterials. The authors apply a numerical method to a new system that takes into account the
 local near-field effects of metamaterial resonators on the dielectric tunability of ferroelectric
 materials in the metamaterial unit cell.

Overall I found the approach described by the authors to be interesting and I expect it will be useful to
 introduce this two-scale convergence method to microwave metamaterial engineering and perhaps to
  metamaterial research in higher frequency ranges as well.

In principle I think this study is publishable, but I would like to see some important clarifying revisions
 made, all are outlined below. After these revisions are adaquately addressed, I expect to be able to
  recommend publication in JAP. The authors need to address the following:


\begin{enumerate}
\item Title/Abstract/ Introduction of Paper: I think the title/abstract and overall introduction of the
 paper is somewhat misleading. These sections should be revised to make clear that the authors are
  presenting a numerical and computational study. As written, it is unclear whether the paper is presenting
   an experimental observation, or a numerical prediction.\\


\rep{It has been clarified in the abstract and introduction that this is a computational and numerical study.}



\item In the introduction, please outline the actual ferroelectric composites in common use in this field.
 It is useful to know what materials this method can be used to model.




\item In the introduction: Local field engineering and field amplification has become more and more useful
 at other frequency ranges, especially in the THz and near-IR, to control the properties of other materials
  (eg. \href{https://doi.org/10.3390/photonics6010022}{Photonics 2019, 6(1), 22}). Could the authors expand
   on some other systems, besides ferro-electric metamaterials, that this method might apply to. For
    instance maybe tunable
    \href{https://journals.aps.org/prl/abstract/10.1103/PhysRevLett.110.217404}{GaAs meta-devices}.

    \rep{We have expanded on this point in the introduction and added possible applications of the model and corresponding references.}


\item In the introduction: The authors mention that there are generally limitations to the current
 numerical and theoretical models used in the field, but do not go into any detail. Could the authors
  expand on this point? What are the limitations?

\item Theory Section, A. In figure 1, are these electrostatics measurements performed by the authors? It is
 unclear based on the figure and surrounding paragraphs. I assume they are as no citation is provided for
  the data. Could the authors discuss the details of how these measurements were made? This would be useful
   for metamaterials researchers who do not work in this frequency range.

\rep{Details about the composition, fabrication and measurements of the samples at DC and microwave frequencies have been added at the beginning of Section II A.}


\item Theory: Equation 3 is normally referred to as Gauss' Law (or it is at least Gauss' Law written in
 terms of V).

 \rep{Reference to Gauss' law has been added before Eq. 3}

\item Figure 3 c) It is unclear what is being plotted here. Is this the tunability, n, defined at the
 beginning of the introduction? If so, please label using previously defined notation.


 \rep{The captions of Figure 3 and other plots have been updated with more details (we are plotting normalized values),
 and more precisions have been added in the main text (section III. A)}




\item Overall the English usage is quite good and the paper is very readable. I did find a few typos
 however and the paper should be proofread again before the next submission. Here are some of the typos I
  found:
\begin{itemize}
  \item Abstract line 7: I believe "wherehas" should be "whereas"

  \item Intro par.2 line 5 "ceramics to" should be "ceramics with"

  \item Intro par 3 line 13 "litterature" should be "literature".

  \item Theory A, last two lines "interrested" should be "interested".
\end{itemize}
\rep{Those typos and a few other ones have been corrected.}

\end{enumerate}

\section{Reviewer \#2}
\subsection{Evaluations}

\begin{itemize}
  \item Does the manuscript present original and timely results that significantly advance the knowledge in
   applied physics: Yes
  \item Does the manuscript report on convincing and rigorous data methods and analysis: Yes
  \item Is the manuscript clearly written in correct English well organized and free from ambiguities: Yes
  \item Is the title descriptive of the contents concise interesting and free of acronyms: No
  \item Does the abstract adequately and clearly describe the contents (problem approach findings) of the
   paper: Yes
  \item Are the figures in the manuscript necessary adequate well presented and clearly labeled: Yes
  \item Is the reference list appropriate: Yes

\end{itemize}


% ~\\
\subsection{Remarks to author(s)}

  The paper is interesting and can be published with very small revisions.

  \begin{enumerate}
  \item Although composites are engineered materials, in my opinion, we cannot call "metamaterial" a
   composite if we do not demonstrate specific frequency-dependent behavior and if the regular repeated
    units characteristic to the metamaterial are not smaller than the corresponding wavelength where we
     found the mentioned property. In this paper there is not an analysis vs frequency of the properties,
      so that your material is a di-phase composite. I suggest to change the title.

\rep{The title has been changed to \textit{Enhanced tunability in ferroelectric composites through local field
enhancement and the effect of disorder}}

  \item You gave experimental data for BST ceramics that were used to derive Landau coefficients. It is
   necessary to give some details of the sample (composition?) and measurements or to cite an experimental
    work.

    \rep{Details about the composition, fabrication and measurements of the samples at DC and microwave frequencies have been added at the beginning of Section II A.}



  \item You may compare the results of your simulations with the results of other studies.


\end{enumerate}



\end{document}
